\documentclass[9pt]{developercv} % Default font size, values from 8-12pt are recommended
\begin{document}

%----------------------------------------------------------------------------------------


%----------------------------------------------------------------------------------------
%   TITLE AND CONTACT INFORMATION
%----------------------------------------------------------------------------------------

\begin{minipage}[t]{0.45\textwidth} % 45% of the page width for name
    \vspace{-\baselineskip} % Required for vertically aligning minipages

    % If your name is very short, use just one of the lines below
    % If your name is very long, reduce the font size or make the minipage wider and reduce the others proportionately
    \colorbox{Sienna}{{\HUGE\textcolor{white}{\textbf{\MakeUppercase{Michael}}}}} % First name

    \colorbox{Sienna}{{\HUGE\textcolor{white}{\textbf{\MakeUppercase{Sendker}}}}} % Last name

    \vspace{6pt}

    {\huge Software Developer} % Career or current job title
\end{minipage}
\begin{minipage}[t]{0.275\textwidth} % 27.5% of the page width for the first row of icons
    \vspace{-\baselineskip} % Required for vertically aligning minipages

    % The first parameter is the FontAwesome icon name, the second is the box size and the third is the text
    % Other icons can be found by referring to fontawesome.pdf (supplied with the template) and using the word after \fa in the command for the icon you want
    \icon{MapMarker}{12}{Tarpon Springs, FL}\\
    \icon{Phone}{12}{+1 (727) 272-3915}\\
    \icon{Envelope}{12}{\href{mailto:m@stdwtr.io}{m@stdwtr.io}}\\
    \icon{Linkedin}{12}{\href{https://linkedin.com/in/michael-sendker}{michael-sendker}}\\
\end{minipage}
\begin{minipage}[t]{0.275\textwidth} % 27.5% of the page width for the second row of icons
    \vspace{-\baselineskip} % Required for vertically aligning minipages

    % The first parameter is the FontAwesome icon name, the second is the box size and the third is the text
    % Other icons can be found by referring to fontawesome.pdf (supplied with the template) and using the word after \fa in the command for the icon you want
    \icon{Globe}{12}{\href{https://standingwater.io}{standingwater.io}}\\
    \icon{Globe}{12}{\href{https://blog.standingwater.io}{blog.standingwater.io}}\\
    \icon{Github}{12}{\href{https://github.com/malan88}{malan88}}\\
    \icon{StackOverflow}{12}{\href{https://stackoverflow.com/story/malan88}{malan88}}\\
\end{minipage}

% this vspace is largely useless and causes me to have trouble fitting everything
%\vspace{0.5cm}

%----------------------------------------------------------------------------------------
%   INTRODUCTION, SKILLS AND TECHNOLOGIES
%----------------------------------------------------------------------------------------

\cvsect{Who I Am}

    I prefer Python, but I wear many hats. I have written code in C\#,
    Python, vanilla JavaScript, and ReactJS. I have dabbled in C, Scheme, C++,
    and Bash. I have built static sites in GatsbyJS, interfaces in WPF, and web
    scrapers and bots in Python. I even built an arduino-based PID controller
    for a popcorn popper to roast my own coffee. But by far most of my time has
    been spent developing a single web technology, consisting of $\approx$
    14,000 lines of code in Python, Flask, and SQLAlchemy, in the pursuit of my
    obsession with literature:
    {\href{https://github.com/malan88/icc}{anno.wiki}}.\\

    All of this should demonstrate that I am a full time learner. I have spent
    enough time learning new things that I have full confidence in my ability to
    learn new things. But most of all, I am a thoughtful programmer. I think
    twice about abstraction or hard coding, I think about tech debt and the way
    users will interact with a feature. When I'm writing code in the present,
    I'm thinking about the future.

%----------------------------------------------------------------------------------------
%   EXPERIENCE
%----------------------------------------------------------------------------------------

\cvsect{Experience}

\begin{entrylist}
    \entry
        {2020 -- Current}
        {\href{https://euler-sci.com}{Euler Sciences LLC}}
        {Contract Developer}
        {
            I worked on a GUI interface for laser system meant to treat skin
            conditions. I also developed the company website in GatsbyJS (and
            did the graphical work myself in Photoshop).
        \\
        \texttt{{\href{https://en.wikipedia.org/wiki/C_Sharp_(programming_language)}{C\#}}}\slashsep
        \texttt{{\href{https://en.wikipedia.org/wiki/Windows_Presentation_Foundation}{Windows Presentation Foundation}}}\slashsep
        \texttt{{\href{https://www.sqlite.org/index.html}{SQLite}}}\slashsep
        \texttt{{\href{https://www.adobe.com/products/photoshop.html}{Photoshop}}}\slashsep
        \texttt{{\href{https://www.gatsbyjs.org/}{GatsbyJS}}}
        }
    \entry
        {2020 -- Current}
        {Healthy Brands, LLC}
        {Contract Developer}
        {
            I've worked on several projects for this company I've:
            \begin{itemize}[leftmargin=*, noitemsep]
                \item built and deployed a multithreaded scraper and alert system for notifying the company of BuyBox loss events on Amazon using a variety of data sources
                \item launched, developed, and maintained {\href{https://ask.fiteyes.com}{ask.fiteyes.com}}, a Django-based question and answer forum built upon {\href{https://github.com/ialbert/biostar-central}{Biostar}}.  Contributed several bug fixes back upstream.
                \item prepared and maintained several VPS servers, including one in Arch Linux
                \item done general systems programming, including Systemd Units and Timers
            \end{itemize}
        \texttt{{\href{https://www.python.org/}{Python}}}\slashsep
        \texttt{{\href{https://www.crummy.com/software/BeautifulSoup/bs4/doc/}{BeautifulSoup4}}}\slashsep
        \texttt{{\href{https://www.djangoproject.com/}{Django}}}\slashsep
        \texttt{{\href{https://www.sqlalchemy.org/}{SQLAlchemy}}}\slashsep
        \texttt{{\href{https://www.postgresql.org/}{PostgreSQL}}}\slashsep
        \texttt{{\href{https://archlinux.org/}{Arch Linux}}}
        }
    \entry
        {2019 -- Current}
        {{\href{https://github.com/Anno-Wiki}{Intertextual Canon Cloud 2}}}
        {Flask/React App}
        {
            This is the second iteration of the ICC (see below), with improved
            data architecture and a single-page application front end based in
            React.  Using Elasticsearch to store texts and PostgreSQL to store
            annotations and general application data allows for faster
            incremental searching. I also wrote more sophisticated text
            processors for breaking text into 100k byte chunks and annotating
            stylistic elements, while stripping all styling from the raw text.
        \\
        \texttt{{\href{https://www.python.org/}{Python}}}\slashsep
        \texttt{{\href{https://flask.palletsprojects.com/en/1.1.x/}{Flask}}}\slashsep
        \texttt{{\href{https://www.sqlalchemy.org/}{SQLAlchemy}}}\slashsep
        \texttt{{\href{https://www.elastic.co/}{Elasticsearch}}}\slashsep
        \texttt{{\href{https://www.postgresql.org/}{PostgreSQL}}}\slashsep
        \texttt{{\href{https://www.docker.com/}{Docker}}}\slashsep
        \texttt{{\href{https://auth0.com/}{Auth0}}}\slashsep
        \texttt{{\href{https://reactjs.org/}{ReactJS}}}
        }
    \entry
        {2017 -- 2019}
        {{\href{https://github.com/Anno-Wiki/icc}{Intertextual Canon Cloud (anno.wiki)}}}
        {Flask App}
        {
            The ICC is a web application designed to allow for collaboratively
            building an exhaustive and definitive repository of annotated
            literature. I designed, developed, deployed, and continue to
            maintain the project solo, managing occasionally to rope in some
            programmer friends for help with various features. It consists of
            $\approx$ 14k lines of code and $\approx$ 100k lines of code churn.
            The backend uses Flask/SQLAlchemy. The frontend uses Jinja2, Sass,
            and VanillaJS. I also had to write several ETL data pipelines for
            processing {\href{https://gutenberg.org}{Project Gutenberg}} texts.
            It is deployed via Heroku. I learned the entire web application life
            cycle on this project, and it continues to teach me. It is currently
            maintained at {\href{https://anno.wiki}{https://anno.wiki}}.
        \\
        \texttt{{\href{https://www.python.org/}{Python}}}\slashsep
        \texttt{{\href{https://flask.palletsprojects.com/en/1.1.x/}{Flask}}}\slashsep
        \texttt{{\href{https://www.sqlalchemy.org/}{SQLAlchemy}}}\slashsep
        \texttt{{\href{https://mariadb.org/}{MariaDB}}}\slashsep
        \texttt{{\href{https://www.elastic.co/}{Elasticsearch}}}\slashsep
        \texttt{{\href{https://docs.pytest.org/en/stable/}{pytest}}}\slashsep
        \texttt{{\href{https://www.heroku.com/}{Heroku}}}
        }
\end{entrylist}

%----------------------------------------------------------------------------------------
%   EDUCATION
%----------------------------------------------------------------------------------------

\cvsect{Education}

\begin{entrylist}
    \entry
        {2008 -- 2010}
        {Philosophy and Classics}
        {Florida State University}
        {Focused on Ancient Philosophy, particularly Aristotle, Philosphy of
        Mind, and Symbolic Logic. Also studied Ancient Greek and Latin.
        President of Philosophy Club (2010)}
    \entry
        {2007 -- 2008}
        {Philosophy}
        {University of South Florida}
        {General introductory classes, Social Philosophy, Critical Thinking}
\end{entrylist}

%----------------------------------------------------------------------------------------
%   ADDITIONAL INFORMATION
%----------------------------------------------------------------------------------------
%
\begin{minipage}[t]{0.2\textwidth}
    \vspace{-\baselineskip} % Required for vertically aligning minipages

    \cvsect{Languages}

        \textbf{English} — native,\\
        \textbf{French} — fluent,\\
        \textbf{Spanish} — rudimentary

\end{minipage}
\hfill
\begin{minipage}[t]{0.7\textwidth}
    \vspace{-\baselineskip} % Required for vertically aligning minipages

    \cvsect{Free Time}

    I am a lifelong learner, constantly finding new things to obsess over. In
    the last 3 years I've learned the intricacies of wine and cheese, learned to
    roast coffee and brew top-tier espresso, become a sophisticated mixologist,
    and learned to speak French.
\end{minipage}
%\hfill
%\begin{minipage}[t]{0.35\textwidth}
%    \vspace{-\baselineskip} % Required for vertically aligning minipages
%
%    \cvsect{Closing Argument}
%
%    I have had enough experience learning new things to have full confidence in
%    my ability to learn new things.
%\end{minipage}

%----------------------------------------------------------------------------------------

\end{document}
